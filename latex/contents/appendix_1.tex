% !TEX root = ../main.tex

\appchapter{符号与标记}\label{chap:symbol}
\addcontentsline{toc}{chapter}{附\quad 录}  


\begin{table}[h]
	\begin{center}
		\begin{tabular}{cc}
			\toprule[1.5pt]
			符号与标记&定义\\
			\midrule[1pt]
            $n$ & 受试者或样本的数量\\
            $d$ & 协变量的维数 \\
            $k$ & 数值模拟时分层随机化使用的因素个数\\
            $B$ & 分层随机化中每层内的样本数量\\
			$\mathbf{X}$& 协变量矩阵\\
			$\mathbf{{W}}$& 分组分配结果向量\\
			$Y_{obs(\mathbf{W})}$或$Y_{obs}$& 观察到的实验结果向量\\
			$\tau$& 平均处理效应(ATE)\\
			$\hat{\tau}$&平均处理效应的估计\\
			$\varphi(\mathbf{x}, \mathbf{W})$&随机化接受准则函数 \\
			$S$&随机化接受准则函数的集合\\
			$P_a$& 可接受随机化的比例\\
            $\bar{\mathbf{X}}_T - \bar{\mathbf{X}}_C$ & 实验组和对照组之间协变量的平均差值\\
			$M$& Mahalanobis 距离\\
			$\varphi_M$& 使用Mahalanobis 距离的随机化可接受准则函数\\
			$v_a$&$\frac{P\left(\chi_{d+2}^2 \leq a\right)}{P\left(\chi_d^2 \leq a\right)}$\\
			$R^2$&实验结果$\mathbf{Y}$和协变量$\mathbf{X}$之间的多重相关系数的平方\\
			\bottomrule[1.5pt]
		\end{tabular}
	\end{center}
\end{table}
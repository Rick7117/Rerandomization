% !TEX root = ../main.tex

\begin{abstract}[zh]
\addcontentsline{toc}{chapter}{摘 \quad 要}
 本文通过理论分析和模拟实验,深入探讨了随机对照试验中的协变量平衡问题及其解决方案。文章首先指出,虽然随机化可以在实验组和对照组之间平衡协变量,但在实际实验中,协变量分组可能存在不平衡现象。为此,文章提出了重随机化的方法,并详细介绍了其执行过程和统计学性质。接着,文章阐述了A/A 测试、区组随机化和分层随机化的具体做法和注意事项,并通过模拟实验,对比了这些随机化实验复杂设计方法的表现。结果显示,这些方法都能保持对平均处理效应的估计的无偏性,且使用重随机化和分层随机化后,平均处理效应估计的方差有很大程度的下降。
\end{abstract}

{
\newfontface{\arial}{Arial}[Scale=0.94]
\ctexset{chapter/format+={\arial}}
\begin{abstract}[en]
\addcontentsline{toc}{chapter}{ABSTRACT}
This article delves into the issue of covariate balance in randomized controlled trials through theoretical analysis and simulation experiments. The paper begins by noting that although randomization can balance covariates between treatment groups, imbalances in covariate grouping may still occur in practical experiments. To address this, the paper proposes the method of rerandomization and provides a detailed account of its implementation process and statistical properties. Furthermore, the article elaborates on the specific practices and design considerations of A/A testing, block randomization, and stratified randomization, comparing the performance of these randomized experimental design methods through simulation experiments. The results indicate that these methods maintain the unbiasedness of the average treatment effect estimates, and the use of rerandomization and stratified randomization significantly reduces the variance of the estimated average treatment effect.

\end{abstract}
}
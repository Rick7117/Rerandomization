% !TEX root = ../main.tex

\chapter{全文总结}


\section{主要结论}

本文主要探讨了随机对照试验中协变量平衡的问题以及通过多种方法改善平衡性的策略。尽管随机化是实验设计中用以平衡实验组和对照组协变量的主要手段,但是,仅依靠随机化并不能保证每一个实验中协变量之间的平衡。为此,提出了重随机化策略,通过设定特定的标准(如Mahalanobis距离超过某阈值)来决定是否需要重随机化,以此来改善协变量的平衡性。使用Mahalanobis距离的重随机化在实际应用中不仅能够保持对平均处理效应的估计的无偏性,还能有效降低组间均值差异的方差,并提高处理效应估计的精度。

除了重新随机化,本文还详细介绍了A/A测试、区组随机化和分层随机化等其他随机实验设计方法。A/A测试通过对实验组和对照组施加相同影响,检验分组的均衡性,从而验证相应A/B测试的效果。区组随机化通过将样本分为指定大小的区组并在区组内进行随机分组,旨在平衡顺序处理引入的偏差。而分层随机化则根据预先确定的分层因素将受试者分层,以在每个层内进行随机分组,从而更好地平衡分层因素在组间的分布。

通过对这些方法的模拟实验与分析,本文展示了重随机化、A/A测试、区组随机化和分层随机化在保持处理效应估计无偏性的同时,能显著降低估计方差。特别是重随机化和分层随机化,它们在提高平均处理效应的估计的精确度方面表现尤为突出。在重随机化实验中,平均处理效应估计的方差随着可接受随机化的比例的增大而增大,同时平均计算时间减小。因此,实验设计者在采用重随机化时需要在估计的准确性和计算时间之间取得平衡。

综上,本研究不仅深化了对随机对照试验中协变量平衡问题的理解,还提供了一系列实用的解决方案,对实验设计者来说,通过采纳这些随机化实验设计的策略,可以有效提升随机对照试验的质量,确保研究结果的准确性和可靠性。

\section{研究展望}
% 更深入的研究……
\subsection{平衡协变量的其他方法}
    使用Mahalanobis距离的重随机化能够很好地对连续的协变量进行平衡,但对离散的协变量并不能直接适用。对于只有少数水平的协变量的情况,区组随机化和分层随机化能够很好地平衡所有协变量。但是对于某个维度有很多水平的协变量的情况,使用分层随机化对每个水平都进行分层会导致层内分到的样本数量不足。此外,如何对连续的协变量使用分层随机化,也值得后续去研究。对于协变量中既有连续和离散的维度,我们认为可以将重随机化和分层随机化结合起来使用:对离散的协变量进行分层,再在每个层次内使用重随机化保证组内分配的均衡。随机化实验设计方法远不止本文所讲述的,读者可以进一步阅读相关文献,了解更多在实验前平衡协变量的随机化实验设计方案。

\subsection{重随机化在顺序分配下的困境}

    在许多医学实验中,招募患者往往是比较困难的。在此类试验中,实验设计者往往无法得到所有协变量后再进行分组,而是在长时间内逐一顺序进行分组。在这种情况下,本文叙述的重随机化方法是不适用的。因而实验设计者需要选择如区组随机化和分层随机化等能够实现顺序分配的随机化实验设计方案。如果实验前协变量没有平衡,通常实验设计者可以使用回归调整等\textbf{事后分析方法}(Post-hoc Methods)。
    
\subsection{高维数据给平衡协变量带来的挑战}
    使用Mahalanobis距离的重随机化会在高维数据的情景下失去随机性。为了解决以上问题,实验设计者可以考虑使用其他距离如$\mathrm{l}_2$距离,构造可接受随机化准则函数。或者,我们可以对使用Mahalanobis距离的可接受随机化函数进行一定的调整,加入正则化项$\lambda I_d$:$M=n p_w\left(1-p_w\right)\left(\overline{\mathbf{X}}_T-\overline{\mathbf{X}}_C\right)^{\top} \left(\operatorname{cov}(\mathbf{x})+\lambda I_d\right)^{-1}\left(\overline{\mathbf{X}}_T-\overline{\mathbf{X}}_C\right)$。上述两种距离的统计学性质在本文中没有进行深入的探讨,读者可以在后续进行深入探究。不止于重随机化,高维数据也会给分层随机化带来挑战:过多的分层因素会导致每个层中分到样本数量不足。此时实验设计者可以考虑两条解决方法:(1) 选择更为关键的维度作为分层的因素;(2) 对每个维度构造合适的线性组合,并将线性组合作为分层的因素。
    


% !TEX root = ../main.tex

\chapter{绪论}

\section{引言}

第一次有记录的临床试验是由James Lind在1747年进行的,旨在找出治疗坏血病的方法\cite{dunn1997james}。经过百余年的发展,随机对照试验在医学、心理学、农业等领域都有着广泛的应用。到20世纪晚期,随机对照试验已经被认为是医学中“理性治疗”的标准方法\cite{meldrum2000brief}。随着技术进步,大规模数据的获取成为可能,随机对照试验在互联网行业大放异彩。在互联网行业中,大量新的交互方式、UI界面、推荐算法和新产品新功能的上线都会通过随机对照试验的验证。

由于随机化方法平衡了所有潜在的混杂因素,因而随机对照试验被视为估计因果效应的“金标准”,然而,如果在某个特定的实验中,随机化产生了在重要协变量上明显不平衡的分组结果。在这种情况下,我们需要回答一个重要的问题:如果我们观察到实验组和对照组在实验结果上的差异,那是来源于我们在实验中引入的处理因素,还是来源于分组后就预先存在的差异?

\section{本文研究主要内容}

针对上述问题,我们希望寻找相应的解决方案, 用以对分组之间的平衡程度进行检测,或是寻找使分组间相对更平衡的实验设计方案。A/A 测试、重随机化、区组随机化和分层随机化等随机化实验设计方法在一定程度上降低了不平衡分组产生的可能性。

本文将从三个部分展开,第一部分将从随机对照试验开始引入,并对重随机化的具体流程和统计学性质进行综述。第二部分着重于对A/A 测试、区组随机化和分层随机化的的具体做法和设计中的注意事项。最后一部分将结合数值实验,对比上述提到的随机化实验设计方法的表现。

\section{本文研究意义}

即使进行了随机化,实验组和对照组之间仍然可能存在差异的可能性比想象中的高。我们考虑每个受试者上的$d$个协变量,如性别、身高、体重等。假设$d$个协变量之间保持独立且设定显著性水平为$\alpha$,那么至少有一个协变量在实验组和对照组之间显示出显著差异的概率为$1-(1-\alpha)^d$。例如,当我们设定$d=14$且显著性水平为5\%时,这个概率可以高达51\%。正如Fisher所说,“大多数实验者在进行随机分配时会震惊地发现,不同分组的分布之间有多么不均等。”\cite{fisher1992arrangement}
因此在随机化实验设计中,仅仅进行随机化往往是不足够的。随机化实验复杂设计在随机对照试验中存在极大的价值,它可以防止试验中对照组和实验组出现较大偏差,产生在除各组接受的干预之外的变量比较性较强的组。本文的目的是介绍随机化实验复杂设计方法,包括随机对照试验的概念和意义,并综述几种随机化实验设计方案,以帮助实验设计者和研究人员更好地设计他们的随机对照试验。


\section{本章小结}

本文主要探讨如何在随机对照试验中有效地平衡实验组和对照组之间的差异。本章首先回顾了随机对照试验的历史和重要性,然后讲述了本文研究的主要内容:如何通过A/A测试、重随机化、区组随机化和分层随机化等方法来降低组间不平衡的可能性;通过数值实验对比这些随机试验设计方法的效果。本文的研究意义在于,通过这项研究,科研人员和实验设计者能更好地进行他们的随机对照试验,得到更有效力的因果推断。

% \section{研究内容}

% 研究的内容和文章讲述顺序

% 本毕业设计将研究重随机化等复杂实验设计的统计学性质。在阅读了相关文献后,我提出了以下可以进行探索研究的课题内容:
% \begin{enumerate}
%     \item 从理论上对重随机化等复杂实验设计的统计学性质进行一定的探究.
%     \item 结合理论与数值模拟探索重随机化及其他非完全随机化的复杂设计在大规模随机化实验中的统计学性质
%     \item 结合数值实验探究并比较重随机化实验、完全随机化实验和完全平衡分组对大规模实验设计的影响
%     \item 对比不同的重随机化算法(利用不同的组间距离判断准则)在可接受随机化的比例等算法计算效率指标上的表现。
%     \item 在协变量高维或样本量较大的情况下探究重随机化面临的挑战和问题。
% \end{enumerate}

% 1 综述和归纳重随机化的基础理论和相关结论。
% 2 了解重随机化算法的评价指标和影响重随机化效果的因素。
% 3 代码复现重随机化算法并通过数值实验得到重随机化对统计分析的影响。
% 4 数值实验:重随机化实验、随机化实验和完全平衡的实验分组的比较,得出相关结论。


% 2.2 随机化分组的不均衡问题

% 2.3 重随机化和其他平衡方法的文献回顾

% 2.4 研究缺口和贡献

% \section{研究意义}

% 重随机化的好处。。。文章提出了什么独特的点



% \section{本章小结}
% 本文……

